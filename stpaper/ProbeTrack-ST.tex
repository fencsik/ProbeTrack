\documentclass[jou,nobf]{apa}

\usepackage{graphicx}

\newcommand{\MSE}{\ensuremath{\mathit{MSE}}}
\headinglevels{one}

\ifapamodeman{\newcommand{\IncludeFigureAPA}[1]{\includegraphics{#1}}}%
             {\newcommand{\IncludeFigureAPA}[1]{\fitfigure{#1}}}%
             

\title{Rapid Recovery of Moving Targets Following Task Disruption}
\threeauthors{David E. Fencsik}{Skyler S. Place}{Todd S. Horowitz}
\threeaffiliations{California State University, East Bay}{Indiana
University}{Brigham and Women's Hospital and Harvard Medical School}

\rightheader{Rapid Recovery in Tracking}
\leftheader{Fencsik, Place, \& Horowitz}
\shorttitle{Rapid Recovery of Moving Targets}
\note{Draft: \today}

\begin{document}

\maketitle

Driving an automobile involves simultaneously performing a variety of
tasks: the driver must maintain control of the vehicle, react to traffic
signals, plan a route, and often carry out these tasks while also listening
to music or carrying on conversations with passengers.  However, all these
tasks are secondary to the basic need to monitor the environment, a
continuously shifting collection of lanes, sidewalks, traffic signs,
pedestrians, and other vehicles.  In order to successfully maneuver through
traffic, we must have cognitive mechanisms that can keep track of a set of
moving objects, focusing on those that are most important and ignoring the
rest, and accomodate brief disruptions from other tasks.  In the present
paper, we focus on the ability of the tracking mechanisms to recover from
disruption.

Tracking mechanisms have been studied in the laboratory through the
multiple object tracking (MOT) task \cite{PylyshynStorm1988}.  In a typical
MOT study, observers are asked to keep track of a subset of stimuli
("targets") as they move independently around a display.  After several
seconds of tracking, all stimuli stop, and the observer indicates whether
or not a probed stimulus was a target; both the accuracy and latency of
each response may be measured as dependent variables.  Observers can
typically track around 4--5 targets out of 8--10 total objects
(\citeauthorNP{PylyshynStorm1988}).  This demonstrates that there is a
mechanism capable of tracking multiple moving objects without maintaining
gaze on them.

Observers can track successfully when tracked stimuli move behind occluders
or disappear briefly \cite{FlombaumScholl2008,SchollPylyshyn1999}, or when
the entire stimulus display disappears temporarily
\cite{AlvarezHorowitz2005,FencsikKlieger2007,HorowitzBirnkrant2006,
  KeanePylyshyn2006}.  In fact, tracking performance is no worse when
interrupted with a brief blank interval of some duration than when
interrupted by another visually demanding task for an equivalent duration
(\citeauthorNP{AlvarezHorowitz2005}).  These results indicate that there
must be some mechanism that enables temporary postponement of the tracking
task during disappearance or disruption
(\citeauthor{HorowitzBirnkrant2006}).  Information relevant to tracking is
stored by this mechanism and is used to reacquire targets following
reappearance.

In the present work, we focus on the timecourse of target reacquisition:
Following the temporary disappearance of all tracked objects, how long does
the visual system take to resume tracking the targets?  We present a series
of experiments using MOT with a gap in tracking.  Following the gap at a
variable delay, one stimulus was probed.  Observers classified the probed
stimulus as a tracked target or an untracked distractor; a target was
probed on half of the trials, and a distractor was probed on the remaining
trials.  We measured the speed and accuracy of responses to the probe as a
function of probe onset latency in order to determine how long it took for
observers to reacquire the targets.

We assume that if the probe onsets after target reacquisition has been
completed for the probed target, then an accurate decision can be made
immediately and reaction time (RT) will not include any time to reacquire
the target.  However, if the probe onsets before the probed target had been
reacquired, then response will be slowed until reacquisition of that target
has been completed.  Therefore, RT will decrease with probe delay until the
probe delay is long enough for all targets to be reacquired; from that
point on, RT should remain constant as a function of probe delay.  Longer
probe delays therefore provide a measure of the baseline RT, the minimum
time required to identify the probed stimulus and respond when all targets
are being tracked.

We predict a systematic relationship between RT and probe delay.
Specifically, RT will decline linearly from probe delays of 0 (i.e., when
the probe is presented as soon as the gap ends), and this decline will have
a slope of -1: This is because each millisecond of probe delay provides
another millisecond to reacquire targets \cite<see for example>{Pashler1994}.
RT will decline until it reaches a baseline level, at which point it will
remain constant as probe delay continues to increase.  The relationship can
be described by a formula with two parameters:
\begin{equation}
  \label{eqn:model} \mathrm{RT} = b + \max\left(r - d, 0\right)
\end{equation}
where $b$ is the baseline reaction time when all targets are being tracked,
$r$ is the time needed to reacquire all targets, and $d$ is the probe
delay.  The variable $d$ is known, leaving $b$ and $r$ as free parameters
that can be fit to observed data to estimate their values: in particular,
the best-fitting value of $r$ provides an estimate of how long it takes to
recover from the task disruption.

Here, we present four experiments, all using the same basic task
(differences between experiments are described below).  The stimuli were
dark gray circles on a light gray background, with diameters of
approximately 1.33 degrees of visual angle.  Trials were self-paced.  Each
trial began with a brief cue phase, during which the targets were drawn in
yellow and distractors in dark gray.  After the cue phase, all stimuli
returned to dark gray and began moving in random directions, with each
stimulus repulsing other stimuli to avoid overlap.  All stimuli continued
to move for 800 to 2400 ms (60 to 180 frames), then disappeared
simultaneously for a 133 ms (10 frame) gap.  Following the gap, all stimuli
reappeared in the locations they would have occupied had they continued
moving along their pre-gap trajectories.  After a variable delay, one
stimulus turned red and the observer had to press one of two keys to
indicate whether this probe disk was a target or distractor.  Observers
tracked four targets among eight total disks.

In Experiment 1, we introduce the basic methodology and observe that target
reacquisition occurs quite rapidly.  Observers responded to probes that
onset 0, 80, 320, or 1280 ms (0, 6, 24, or 96 frames) after the gap.  Probe
onset delay was randomized within block.  The design resulted in 50 trials
per level of probe delay (25 target probes and 25 distractor probes).
Observers included 6 females and 2 males ranging in age from 22--40 years
($M = 29.1$ years).  We computed each observer's median RT from correct
responses, collapsing across probe types.  The open circles in Figure
\ref{fig:exp1} indicate average median RT as a function of probe delay; the
filled squares and dashed line indicate average $d'$
\cite{MacMillanCreelman2005}.  We conducted separate repeated measures
ANOVAs on median RT and $d'$ as a function of probe delay.  There was a
reliable effect of probe delay on median RT; $F(4,28) = 9.34, \MSE = 1256,
p < .001$.  The main effect of probe delay on $d'$ was marginally reliable;
$F(4, 28) = 2.27, \MSE = 0.1914, p = .087$.  The error bars indicate 95\%
confidence intervals based on the within subject error
\cite{MassonLoftus2003}.
\begin{figure}
  \centering
  \IncludeFigureAPA{an/pb1}
  \caption{Performance in Experiment 1 as a function of probe delay.}
  \label{fig:exp1}
\end{figure}

To further characterize the results, we fit the model from
Equation \ref{eqn:model} to each observer's median RT by probe delay
function.  We used a gradient descent algorithm to find the values for each
parameter, $b$ and $r$, that best fit the observed median RTs.  The
predicted median RTs based on the average parameter values are plotted as
the solid line in Figure \ref{fig:exp1}.  The averaged function predicts
the observed data well, with $R^2 = .88$.  The average estimated
reacquisition time was 83 ms, with a 95\% confidence interval (CI) spanning
38--128 ms.  This suggests that observers tended to recover from the
disruption in tracking by about 80 ms following the end of the gap.

The range of the estimate for recovery time was rather large in Experiment
1.  We conducted Experiment 2 in order to generate a more precise estimate
of reacquisition time, adding probe delays to cover the 95\% CI estimated
in Experiment 1.  Probes appeared 0, 40, 80, 120, 160, or 1280 ms (0, 3, 6,
9, 12, or 96 frames) after gap offset.  Observers included 7 females and 1
male ranging in age from 24--45 years ($M = 31.9$ years).  The results of
this experiment are plotted in Figure \ref{fig:exp2}, which is analogous to
Figure \ref{fig:exp1}.  There was a reliable effect of probe delay on
median RT; $F(5, 35) = 2.99, \MSE = 1423, p = .024$.  However, there was no
reliable effect of probe delay on $d'$; $F(5, 35) = .57, \MSE = 0.25, p =
.719$.  The average estimated reacquisition time, based on the best fit of
Equation \ref{eqn:model} to each observer's results, was 44 ms, with a 95\%
CI spanning 18--69 ms.
\begin{figure}
  \centering
  \IncludeFigureAPA{an/pb2}
  \caption{Performance in Experiment 2 as a function of probe delay.}
  \label{fig:exp2}
\end{figure}

One possible explanation for the fast recovery times is that no recovery
was necessary; that is, the 133-ms gap was too brief to trigger
task-postponement mechanisms.  To test this possibility, we conducted a
third experiment in which we varied gap duration, including gaps of 133,
307, and 507 ms (10, 23, and 38 frames). In previous experiments, we have
shown that performance with these longer gaps is equivalent to performance
with a secondary task, suggesting that such gaps do trigger
task-postponement mechanisms (Alvarez et al., 2005).  If a 133 ms gap does
not involve task postponement, then we would expect gap duration to effect
the RT by probe delay function in Experiment 3, and to lengthen the
estimated reacquisition time.  Observers included 5 females and 3 males
ranging in age from 19--50 years ($M = 29.4$ years). The results are
plotted in Figure \ref{fig:exp3}.  There was no effect of gap duration on
median RT; $F(2, 14) = 0.054, \MSE = 36315, p = .947$.  Median RT did
decline with probe delay; $F(7, 49) = 6.71, \MSE = 1385, p < .001$.  There
was a marginally reliable interaction between probe delay and gap duration,
but this was caused by a single unusual data point in the 320-ms probe
delay and 307-ms gap duration condition.  Aside from this one point,
post-gap performance was identical across gap durations, suggesting that
the same task-postponement mechanisms that enable tracking through a 307-
or 507-ms gap also function during a 133-ms gap.  To obtain a reliable
estimate of reacquisition time, we fit the model to the data from each
observer collapsed across gap duration.  The average estimated
reacquisition time was 44 ms, with a 95\% CI spanning 23--65 ms.
\begin{figure}
  \centering
  \IncludeFigureAPA{an/pb3a}
  \caption{Performance in Experiment 3 as a function of probe delay and gap
    duration.}
  \label{fig:exp3}
\end{figure}

The first three experiments suggest that targets are reacquired rapidly
following resumption of tracking.  In Experiment 4, we tested whether the
number of targets affects reacquisition time.  Observers tracked 2 targets
instead of 4.  If target recovery is serial, then reacquisition time should
be shorter than in previous experiments.  It may also take less time to
respond overall with fewer targets, but this will manifest itself as a
reduction in average RT, not in reacquisition time.  Observers included 6
females and 4 males ranging in age from 18--37 years ($M = 23.4$ years; one
observer's age was not recorded).  There was a reliable effect of probe
delay on median RT; $F(3, 27) = 5.54, \MSE = 529, p < .01$.  However, there
was no effect of probe delay on $d'$; F(3, 27) < 1.  In a separate
analysis, we compared Experiment 4 to Experiment 2.  There was a reliable
difference between experiments ($F(1, 16) = 13.2, \MSE = 68647, p = .002$)
and an effect of probe delay ($F(3, 48) = 9.63, \MSE = 830, p < .001$), but
no reliable interaction ($F(3, 48) = 1.95, \MSE = 830, p = .13$).  The
difference between experiments reflects the fact that RT was shorter in
Experiment 4 ($M = 513$ ms) than in Experiment 2 ($M = 733$ ms).  The
average estimated reacquisition time was 43 ms, with a 95\% CI spanning
23--64 ms, which is effectively identical to that observed in previous
experiments.  Thus, it appears that 2 targets take just as long to recover
as 4 targets.

The experiments presented above show that recovery of moving targets after
a brief disappearance occurs rapidly and in parallel.  Around 40 ms after
the tracked stimuli reappeared, observers could distinguish targets from
distractors as quickly as if the stimuli had never disappeared, suggesting
that the tracking mechanisms were in the same state they had been in prior
to disappearance.

These results seem to differ from decades of research on task switching,
which tend to find RT effects of switching tasks, the so called {\em switch
  cost}, on the order of 100--200 ms (e.g.,
\citeNP{AllportStyles1994,RogersMonsell1995,RubinsteinMeyer2001}; see
citeNP{Monsell2003}, for a recent review).  However, the range of these
overlap with the present results (e.g., see Table 1 in
\citeNP{Altmann2007}), and is entirely consistent with the switch costs
obtained in Experiment 4 of \citeA{RogersMonsell1995}, when there was no
interference between tasks and observers were given 600 ms or more to
prepare for the upcoming task.  The present results come from a situation
in which there is no interference from a secondary, so are entirely
consistent with results reported from previous studies of task switching.

Solve problem of just being a surprise: results of asynchtrack
\cite{HorowitzBirnkrant2006} are relevant.

We propose that these findings generally indicate that any visual
information stored off-line during task postponement will be available
again rapidly after the task is resumed.  This differs from other findings
of task-switching costs that last for several hundred milliseconds when
people must switch from one task to another (e.g.,
\citeNP{AllportStyles1994,RogersMonsell1995,RubinsteinMeyer2001}; see
\citeNP{Monsell2003}, for a recent review).  To perform those tasks
successfully, all the processes relevant to performing one task---including
all the knowledge relevant to the stimuli, stimulus-response mappings, and
response preparation---must be removed from memory and replaced with those
of a different task.  The present results suggest that the cost of
reloading the visual information relevant to one task is quite small.  It
may be other aspects of a task, or the act of unloading another task from
memory, that imposes the higher switching costs observed in other studies.

\bibliography{../../biblio/references}

\end{document}
