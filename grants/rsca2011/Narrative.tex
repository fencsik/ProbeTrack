\documentclass[12pt]{article}
\usepackage[hmargin=1in,vmargin=1in]{geometry}
\usepackage[T1]{fontenc}
\usepackage{textcomp}
\usepackage{times}
\usepackage{indentfirst}
\usepackage{setspace}
%\doublespacing
\setstretch{1.85}
\newcommand{\MySection}[2]{%
  \normalfont\normalsize\noindent{\bf#1.\hspace{1em}#2}}
\setlength{\parindent}{0.5in}

\begin{document}

\MySection{A}{Summary Description}

% Overview

Under this proposal, I seek support for two goals.  The first is to write a
manuscript for professional publication based on a set of completed
experiments.  The second is to begin planning and developing a set of
follow-up experiments to advance my research.

% My Background and Topic

My background is in cognitive psychology.  This field seeks to analyze
human thought by breaking it into a set of separate systems, each of which
process information in some way.  The focus of my research is on attention
and short-term memory, the processes that select relevant information and
store it briefly while it is relevant to the task at hand.

% The project

The current proposal seeks to advance my research on how attention switches
between different sources of information.  We do this regularly in our
daily activities.  For example, when we are driving, we regularly look away
from the traffic in front of us to check a mirror or glance at the
dashboard.  My colleagues and I have developed a technique that allows us
to precisely measure the recovery time following such a shift in attention.
We have conducted many experiments to test the validity and reliability of
the technique, and have found that it takes roughly 40 milliseconds (0.04
seconds) to recover after an attentional shift.

% This specific proposal

Under this proposal, I will write a manuscript based on the completed
experiments and develop the next stage of this project.  The experiments
were intended to test the technique and establish a baseline measure of
recovery time.  The results need to be published to communicate the
technique to other researchers and establish it in the literature. The next
stage of this project will involve applying the technique to situations in
which other distracting tasks being performed at the same time as the MOT
task.

This next stage is relevant to real-world tasks.  To continue the earlier
example, while driving and switching attention between various important
sources of information, many people also pay attention to less important
things such as navigation systems or cell phone conversations.  A great
deal of research has established that such distracting tasks can be
dangerous, but nobody has yet established how they affect attention.  My
technique presents a means to investigate a specific aspect of attention
that may be affected by distracting tasks.

\MySection{B}{Methods, Procedures, and Student Involvement}

This project

The technique used to measure attention is based on a standard experimental
task called the multiple object tracking (MOT) task.  In it, a set of
identical objects move randomly around a computer display.  People must
keep track of a subset of those objects called targets, despite the fact
that targets look identical to all other stimuli.  This is a demanding task
because any lapse in attention will result in lost targets.

In our version of the technique, all the objects disappear briefly (for
about a tenth of a second), then reappear.  We can determine how long it
takes people to recover from the disappearance by measuring their
performance at various intervals after reappearance.

\MySection{C}{Timetable}

\MySection{D}{IRB Status}

The experiments in this proposal were conducted on human subjects.  The
project has been reviewed by the CSUEB IRB and deemed exempt from review
(IRB project title: ``Mechanisms of Visual Attention and Short-Term
Memory'').  The project was originally reviewed in July, 2008, and was most
recently renewed on July 14, 2010.

\MySection{E}{Relationship to Prior and Future Work}

\end{document}
