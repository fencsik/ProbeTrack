\documentclass[12pt]{article}
\usepackage[hmargin=1in,vmargin=1in]{geometry}
\usepackage[T1]{fontenc}
\usepackage{textcomp}
\usepackage{times}
\usepackage{indentfirst}
\usepackage{setspace}
%\doublespacing
\setstretch{1.85}
\newcommand{\MySection}[2]{%
  \normalfont\normalsize\noindent{\bf#1.\hspace{1em}#2}}
\setlength{\parindent}{0.5in}

\begin{document}

\MySection{A}{Summary Description}

% Overview

I seek assigned time to write a manuscript for professional publication
based on a set of completed experiments.  The proposed project will allow
me to finish one phase of a longer project and lay the groundwork for the
next phase.

% My Background and Topic

My background is in cognitive psychology.  Cognitive researchers study
human thought by breaking it into a set of separate systems, each
responsible for processing information in some way.  The focus of my
research is on attention and short-term memory.  These processes select
important information and store it briefly while it is relevant to the task
at hand.

% The project

Under the current proposal, I plan to advance a research project in which I
investigate how attention switches between different sources of
information.  We do this regularly in our daily activities.  For example,
when we are driving, we regularly look away from the traffic in front of us
to check a mirror or glance at the dashboard.  My colleagues and I have
developed a technique that allows us to precisely measure the recovery time
following such a shift in attention.  We have conducted many experiments to
test the validity and reliability of the technique, and have found that it
takes roughly 40 milliseconds (0.04 seconds) to recover after an
attentional shift.

% This specific proposal

I will use the assigned time from this proposal to write a manuscript based
on the completed experiments.  The experiments were designed to test the
technique and to establish a baseline measure of recovery time.  The
results were presented at a conference last year and generated interest.  I
have not yet had time to write a paper based on this work due to my
teaching load; reducing that load will enable me to write a manuscript.

\MySection{B}{Methods, Procedures, and Student Involvement}

The proposed project focuses on writing, not running studies.
Nevertheless, I will summarize the methods used in the completed
experiments.  These studies used standard experimental psychology
procedures.  We developed a computerized task similar to a video game, then
measured subjects' performance while they completed the task.

The technique we used is based the multiple object tracking (MOT) task.  In
MOT, a set of identical objects move randomly around a computer display.
People must keep track of a subset of those objects called targets, despite
the fact that targets look identical to all other stimuli.  This is a
demanding task because any lapse in attention will result in lost targets.
In our version of the MOT task, all the objects disappear for about a tenth
of a second, then reappear.  We can determine how long it takes people to
recover from the disappearance by measuring their performance at multiple
intervals after reappearance.

CSUEB students have been involved at several stages of this project.
Students in my PSYC 3100 section in Fall 2008 helped to run a pilot version
of this experiment.  Melanie Johnson, who graduated last year with a
B.A. in Psychology from CSUEB, collected much of the data and will be a
co-author on the manuscript.  Future work on this project also will be
completed with the assistance of CSUEB Psychology students.

\MySection{C}{Timetable}

I will develop a final control experiment in Summer 2011 and run it during
Fall quarter.  I plan to complete the manuscript in the Winter quarter of
2012.  I anticipate the following approximate timetable for this proposal:

\begin{center}
  \begin{singlespace}
    \begin{tabular}{ll}\hline
      January & Finish analyzing all data \\
      January--February & Write draft \\
      February--March & Revisions and feedback from co-authors \\
      End of March & Submit to journal \\\hline
    \end{tabular}
  \end{singlespace}
\end{center}

\MySection{D}{IRB Status}

The experiments in this proposal were conducted on human subjects.  The
project was reviewed by the CSUEB IRB in December, 2007, and deemed exempt
from review (IRB project title: ``Mechanisms of Visual Attention and
Short-Term Memory'').  The most recent continuation review is valid until
July 14, 2011.

\MySection{E}{Relationship to Prior and Future Work}

The manuscript I plan to write will complete work that was originally begun
during my postdoctoral research fellowship and finished at CSUEB.
Publishing this work will advance my scholarship and academic career, and
will finish off one phase of the project and allow me to focus my efforts
on developing the next phase.

In the next phase, I plan to apply the technique to multitasking
situations.  To continue the earlier example, while driving and switching
attention between various important sources of information, people often
are distracted by less critical things such as navigation systems or cell
phone conversations.  A great deal of research has established that such
distracting tasks can be dangerous, presumably because they impair
attention.  However, nobody has yet determined exactly what aspect of
attention is affected by distraction.  My technique presents a means to
investigate the effects of distracting tasks on a specific aspect of
attention.

\end{document}
